\documentclass{beamer}
\usetheme{CambridgeUS}
\usepackage[slantfont,boldfont]{xeCJK}
\usepackage{fontspec}
\setCJKmainfont{楷体}
\setCJKsansfont{楷体}
\usepackage{graphicx}
\usepackage{ulem}
\usepackage{tikz}
\usepackage{geometry}
\usepackage{ulem}
\geometry{left=1.0cm,right=1.0cm}


\begin{document}

\title{如何判定半导体的导电类型}
\author{计算机科学与技术系 \\ 孙际儒}
\frame{\titlepage}

\begin{frame}
  \frametitle{Outline}
  \tableofcontents
\end{frame}

\section{技术栈:以python为核心}
\begin{frame}
  \begin{itemize}
    \item 游戏引擎:Ren'Py
    \item 剧本处理脚本:Python
  \end{itemize}
\end{frame}

\section{剧情设计:南大+碳中和}
\begin{frame}
  
\end{frame}

\section{开发方法:敏捷开发}
\begin{frame}{敏捷开发的意义}
  \begin{itemize}
    \item 当你以300km/小时的速度飞奔的时候
    \item 敏捷就显得至关重要
    \item 因为这是你闪避前方障碍物唯一的保障
  \end{itemize}
\end{frame}

\begin{frame}{实现敏捷开发}
  \begin{itemize}
    \item 创建有效的用户故事
    \item 多个分支单独测试
    \item 每周生成一个试玩版本
  \end{itemize}
\end{frame}

\begin{frame}{测试工作安排}
  \begin{itemize}
    \item 各个分支单独测试
    \item 合并分支整体测试
    \item 从第八个教学周开始,启动内测计划
  \end{itemize}
\end{frame}

\end{document}